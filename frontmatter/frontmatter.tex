% we include the glossary here (frontmatter is included with \input, so this command is as if it was in main.tex)
%All acronyms must be written in this file.
\newacronym{SOC}{SOC}{Security Operations Center}
\newacronym{24/7}{24/7}{twenty-four seven}
\newacronym{ML}{ML}{Machine Learning}
\newacronym{AI}{AI}{Artificial Intelligence}
\newacronym{CND}{CND}{Computer Network Defense}
\newacronym{CSIRT}{CSIRT}{Computer Security Incident Response Team}
\newacronym{IPS}{IPS}{Intrusion and Prevension System}
\newacronym{IDS}{IDS}{Intrusion and Detection System}
\newacronym{EDR}{EDR}{Endpoint Detection and Response}
\newacronym{SIEM}{SIEM}{Security Information and Event Management}
\newacronym{APT}{APT}{Advanced Persistent Threats}
\newacronym{RF}{RF}{Random Forest}
\newacronym{RL}{RL}{Reinforcement Learning}
\newacronym{DNN}{DNN}{Deep Neural Network}
\newacronym{CNN}{CNN}{Convolutional Neural Network}
\newacronym{RNN}{RNN}{Recurrent Neural Network}
\newacronym{LSTM}{LSTM}{Long Short-Term Memory}
\newacronym{GRU}{GRU}{Gated Recurrent Unit}
\newacronym{DLM}{DLM}{Deep Learning Model}
\newacronym{DTC}{DTC}{Decision Tree Classifier}
\newacronym{DTM}{DTM}{Decision Tree Model}



\frontmatter % Use roman page numbering style (i, ii, iii, iv...) for the pre-content pages

\pagestyle{plain} % Default to the plain heading style until the thesis style is called for the body content

%----------------------------------------------------------------------------------------
%	TITLE PAGE
%----------------------------------------------------------------------------------------

\maketitlepage


%----------------------------------------------------------------------------------------
%	STATEMENT of INTEGRITY
%----------------------------------------------------------------------------------------
\integritystatement

%----------------------------------------------------------------------------------------
%	DEDICATION  (optional)
%----------------------------------------------------------------------------------------
%
%\dedicatory{For/Dedicated to/To my\ldots}
\begin{dedicatory}
    To my family and friends.
\end{dedicatory}

%----------------------------------------------------------------------------------------
%	ABSTRACT PAGE
%----------------------------------------------------------------------------------------

\begin{abstract}

% here you put the abstract in the main language of the work.

Threats to Cybersecurity have grown ever more sophisticated over the years, making Security Operations Centres (SOC) more important than ever. 
The topic of this dissertation is the application of Machine Learning (ML) to automate the triage of security alerts, addressing issues such as alert fatigue and false positives. 
The research proposes leveraging ML models to improve existing Security Information and Event Management (SIEM) systems by classifying alerts, prioritizing actionable threats, and thus enabling fast detection and response times.

\end{abstract}

\begin{abstractotherlanguage}
% here you put the abstract in the "other language": English, if the work is written in Portuguese; Portuguese, if the work is written in English.

A crescente complexidade e frequência das ameaças cibernéticas têm destacado a importância dos Centros de Operações de Segurança (SOC) na defesa de organizações contra incidentes de segurança. 
Este trabalho investiga a aplicação de técnicas de Aprendizagem Automática (ML) para automatizar a triagem de alertas de segurança, reduzindo os desafios enfrentados pelos analistas, como a fatiga causada pelo alto volume de alertas e a alta taxa de falsos positivos. 
A integração de modelos de ML com sistemas de Gestão de Informações e Eventos de Segurança (SIEM) procura melhorar a eficiência da classificação de alertas, priorizar ameaças críticas e reduzir os tempos médios de detecção e resposta.

Este trabalho inclui uma avaliação dos principais sistemas SIEM, ferramentas de gestão de tickets e frameworks de ML, destacando as suas vantagens e limitações em ambientes SOC. 

\end{abstractotherlanguage}

%----------------------------------------------------------------------------------------
%	ACKNOWLEDGEMENTS (optional)
%----------------------------------------------------------------------------------------

\begin{acknowledgements}

I would like to express my deepest gratitude to my advisor, Professor Jorge Pinto Leite at ISEP, for his invaluable guidance, support, and dedication throughout this project. 
His insights and expertise were crucial in shaping this work and guiding me through the challenges of research and development.

I also extend my heartfelt thanks to my supervisor, Gonçalo Amaro, at ArtResilia, for accepting my idea and providing the opportunity to implement it within the organization. 
His continuous support, advice, and willingness to share his knowledge were essential in bringing this project to fruition.

Their combined mentorship and encouragement were instrumental in the completion of this dissertation, and I am profoundly grateful for their time and efforts.

\end{acknowledgements}

%----------------------------------------------------------------------------------------
%	LIST OF CONTENTS/FIGURES/TABLES PAGES
%----------------------------------------------------------------------------------------

\tableofcontents % Prints the main table of contents

\listoffigures % Prints the list of figures

\listoftables % Prints the list of tables

% \iflanguage{portuguese}{
% \renewcommand{\listalgorithmname}{Lista de Algor\'itmos}
% }
% \listofalgorithms % Prints the list of algorithms
% \addchaptertocentry{\listalgorithmname}


% \renewcommand{\lstlistlistingname}{List of Source Code}
% \iflanguage{portuguese}{
% \renewcommand{\lstlistlistingname}{Lista de C\'odigo}
% }
\listoflistings % Prints the list of listings (programming language source code)
\addchaptertocentry{\lstlistlistingname}


%----------------------------------------------------------------------------------------
%	ABBREVIATIONS
%----------------------------------------------------------------------------------------
%\begin{abbreviations}{ll} % Include a list of abbreviations (a table of two columns)
%%\textbf{LAH} & \textbf{L}ist \textbf{A}bbreviations \textbf{H}ere\\
%%\textbf{WSF} & \textbf{W}hat (it) \textbf{S}tands \textbf{F}or\\
%\end{abbreviations}

%----------------------------------------------------------------------------------------
%	SYMBOLS
%----------------------------------------------------------------------------------------

% \begin{symbols}{lll} % Include a list of Symbols (a three column table)

% $a$ & distance & \si{\meter} \\
% $P$ & power & \si{\watt} (\si{\joule\per\second}) \\
% %Symbol & Name & Unit \\

% \addlinespace % Gap to separate the Roman symbols from the Greek

% $\omega$ & angular frequency & \si{\radian} \\

% \end{symbols}



%----------------------------------------------------------------------------------------
%	ACRONYMS
%----------------------------------------------------------------------------------------

\newcommand{\listacronymname}{List of Acronyms}
\iflanguage{portuguese}{
\renewcommand{\listacronymname}{Lista de Acr\'onimos}
}

%Use GLS
\glsresetall
\printglossary[title=\listacronymname,type=\acronymtype,style=long]

%----------------------------------------------------------------------------------------
%	DONE
%----------------------------------------------------------------------------------------

\mainmatter % Begin numeric (1,2,3...) page numbering
\pagestyle{thesis} % Return the page headers back to the "thesis" style
