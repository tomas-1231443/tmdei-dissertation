\chapter{Conclusion and Future Work}
\label{chap:Chapter5}

\section{Future Work}

While the current solution shows what can be done in automating the security alert triage process, there remain several areas for future improvement and refinement. 

One promising direction for future work involves the expansion of the training dataset. 
While the current model uses a historical dataset to train the RF model, it could benefit from a broader range of data sources. 
Integrating real-time data streams and continuously feeding the system with updated alerts will allow the models to adapt more quickly to evolving attack vectors. 
The inclusion of external sources, such as threat intelligence feeds and behavioral data, may help improve the model's ability to detect and classify previously unseen threats.

Furthermore, enhancing the interpretability of the models, particularly the RL component, would be beneficial. 
While the current system provides high adaptability and can reduce false positives, the underlying decision-making process of the RL model remains relatively opaque. 
Implementing techniques such as SHAP (SHapley Additive exPlanations) or LIME (Local Interpretable Model-Agnostic Explanations) could help increase transparency, allowing analysts to better understand why certain predictions are made and improving trust in the system.

In terms of model optimization, future work could explore hyperparameter tuning for both the RF and RL models. 
While the current version uses a well-optimized set of parameters, there remains room for improvement through more advanced techniques such as automated machine learning (AutoML) or Bayesian optimization. 
These approaches could help identify the optimal configuration for both models, enhancing their overall performance in terms of accuracy, recall, and the ability to generalize across diverse security scenarios.

Finally, the system's real-time performance in a live production environment could be further explored. 
Currently, the RL model is expected to improve over time based on feedback. 
However, integrating continuous model evaluation and performance tracking will allow for more dynamic model updates. 
A feedback loop that accounts for the effectiveness of predictions in real-world conditions could lead to faster model adaptation, thereby improving the accuracy and efficiency of the system in managing alerts. 
Regularly monitoring the system's real-world performance will help identify any gaps and enable proactive updates.

So while the solution has proven effective in its current form, future advancements in data collection, model training, interpretability, and real-time performance tracking will be critical to ensuring that the bot remains adaptable to emerging threats and continues to provide accurate predictions in complex, dynamic environments.

\section{Conclusion}
