\chapter{Conclusion and Future Work}
\label{chap:Chapter5}

This chapter presents the main conclusions drawn from the research and proposes several directions for future work that could further enhance and extend the developed solution. 
The discussion reflects on the accomplishments of the current system while identifying areas for potential improvement and continued evolution.

\section{Future Work}

While the current solution shows what can be done in automating the security alert triage process, there remain several areas for future improvement and refinement. 

First, one of the main limitations observed in the live environment was the system's reduced ability to generalize to underrepresented or noisy alert categories. 
Future iterations should therefore focus on enhancing dataset diversity and volume. 
In particular, having larger and more balanced datasets, either through long-term data accumulation or synthetic data generation, could significantly improve the mode's ability to generalize in production settings.

Secondly, the RL component requires a longer feedback window to become fully effective. 
Extending the live testing period and accumulating more high-quality, consistent feedback will allow the RL model to adapt its policy more accurately. 
Additionally, implementing stronger input validation and preprocessing for feedback, particularly to handle inconsistencies in label formatting—would improve the success rate of training tasks and reduce noise during learning.

Model explainability also remains an open area. 
While the system functions well as a black-box predictor, integrating interpretable ML tools such as SHAP or LIME could help analysts understand the rationale behind predictions. 
This would be especially important for justifying decisions in high priority cases and for refining model behavior in edge scenarios.

From an engineering perspective, future work could also include fine-grained monitoring of per-class performance in real time, enabling automated retraining triggers when class-specific accuracy drops below thresholds. 
Moreover, improvements in resource handling—such as memory-efficient training procedures and container orchestration—would further strengthen the system's robustness under constrained environments.

Finally, exploring ensemble strategies that dynamically balance the contribution of the RF and RL components based on confidence levels or historical performance could lead to better hybrid predictions. 
This would help the system adaptively prioritize models depending on alert complexity and feedback richness.

\section{Conclusion}

This work presented the design, implementation, and evaluation of an intelligent system for automating security alert triage, integrating supervised learning via a RF classifier with a RL agent capable of evolving based on analyst feedback. 
The system was successfully deployed within a live SOC environment, where it processed real alerts, adapted to feedback, and maintained integration with existing tools such as IBM QRadar SOAR.

In controlled local testing, the RF model demonstrated excellent classification accuracy and recall across both priority and taxonomy labels. 
In the live environment, the hybrid system showed initial promise, successfully processing hundreds of alerts while adapting to real-world constraints such as inconsistent data and limited resources. 
Although the RL model had not yet reached its full potential within the short evaluation window.

The successful deployment of a containerized architecture, the integration with SOAR platforms, and the implementation of asynchronous training workflows via Celery and Flower confirm the operational feasibility and practical value of augmenting SOC operations with intelligent, adaptive models.

While challenges remain—particularly in generalizing to minority classes and scaling learning over time, the progress demonstrated in this work offers a strong foundation for future development. With continued refinement, broader datasets, and longer deployment times, this approach has the potential to significantly reduce analyst workload, increase triage accuracy, and enhance the responsiveness of SOC operations to emerging threats.

This thesis concludes with the conviction that combining classical machine learning with reinforcement learning in real-time SOC workflows is feasible and promising towards a smarter, more autonomous cybersecurity system.

